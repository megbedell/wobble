\documentclass[twocolumn]{aastex62}

\usepackage{graphicx}				% Use pdf, png, jpg, or eps§ with pdflatex; use eps in DVI mode
								% TeX will automatically convert eps --> pdf in pdflatex	
\usepackage{xcolor}
\usepackage[sort&compress]{natbib}
\usepackage[hang,flushmargin]{footmisc}
\usepackage[counterclockwise]{rotating}

% units macros
\newcommand{\unit}[1]{\mathrm{#1}}
\newcommand{\km}{\unit{km}}
\newcommand{\m}{\unit{m}}
\newcommand{\cm}{\unit{cm}}
\newcommand{\s}{\unit{s}}
\newcommand{\kms}{$\km\,\s^{-1}$}
\newcommand{\cms}{$\cm\,\s^{-1}$}
\newcommand{\ms}{$\m\,\s^{-1}$}
\newcommand{\ang}{\text{\normalfont\AA}}

% text macros
\newcommand{\documentname}{\textsl{Article}}
\newcommand{\sectionname}{Section}
\newcommand{\todo}[1]{\textcolor{red}{#1}}  % gotta have \usepackage{xcolor} in main doc or this won't work
\newcommand{\acronym}[1]{{\small{#1}}}
\newcommand{\project}[1]{\textsl{#1}}
\newcommand{\foreign}[1]{\textsl{#1}}
\newcommand{\code}[1]{\texttt{#1}}
\newcommand{\HARPS}{\project{\acronym{HARPS}}}
\newcommand{\HIRES}{\project{\acronym{HIRES}}}
\newcommand{\RV}{\acronym{RV}}
\newcommand{\EPRV}{\acronym{EPRV}}
\newcommand{\CRLB}{\acronym{CRLB}}
\newcommand{\wobble}{\code{wobble}}
\newcommand{\TF}{\code{TensorFlow}}
\newcommand{\python}{\code{python}}

\newcommand{\Mdwarf}{Barnard's Star}



\begin{document}
\graphicspath{ {../figures/} }
\DeclareGraphicsExtensions{.pdf,.eps,.png}

\title{\textsc{\wobble: a data-driven method for precision radial velocities}}

\author[0000-0001-9907-7742]{Megan Bedell}
\affiliation{Center for Computational Astrophysics, Flatiron Institute, 162 Fifth Ave., New York, NY 10010, USA}

\author[0000-0003-2866-9403]{David W. Hogg}
\affiliation{Center for Computational Astrophysics, Flatiron Institute, 162 Fifth Ave., New York, NY 10010, USA}
\affiliation{Center for Cosmology and Particle Physics, Department of Physics, New York University, 726 Broadway, New York, NY 10003, USA}
\affiliation{Center for Data Science, New York University, 60 Fifth Ave, New York, NY 10011, USA}
\affiliation{Max-Planck-Institut f\"ur Astronomie, K\"onigstuhl 17, D-69117 Heidelberg}

\author{Daniel Foreman-Mack\'{e}y}
\affiliation{Center for Computational Astrophysics, Flatiron Institute, 162 Fifth Ave., New York, NY 10010, USA}

\author{Benjamin T. Montet}
\affiliation{Department of Astronomy and Astrophysics, University of Chicago, 5640 S. Ellis Ave, Chicago, IL 60637, USA}

\correspondingauthor{Megan Bedell}
\email{E-mail: mbedell@flatironinstitute.org}

\begin{abstract}
% Context
Telluric absorption features in stellar spectra provide particular challenges to extreme-precision radial velocity (\EPRV) exoplanet surveys. 
Large amplitude features are often masked in data analysis pipelines, reducing the number of observable spectral features which can be used to measure an \RV\ shift.
Low-amplitude features that are hard to see even in high signal-to-noise spectra may be an important contributor to the overall \RV\ noise budget.
% Aims
Here we propose a data-driven method to simultaneously extract precise \RV s and infer the underlying stellar and telluric spectra using a fully linear model. 
We implement this method in \wobble, an open-source \python\ package which uses \TF\ in its first non-machine-learning application to astronomical data. 
In this work, we demonstrate the performance of \wobble\ on archival \HARPS\ spectra.
% Results
We recover the canonical exoplanet 51 Pegasi b at \todo{improved accuracy?} relative to the standard \HARPS\ pipeline \RV s, and we achieve a precision of \todo{xx} \cms\ on the \RV\ standard M dwarf \Mdwarf. 
We also present a detailed telluric spectrum derived from these data. 
Our method may be of particular interest for future red-optimized spectrographs aiming to detect and characterize M dwarf host stars, where telluric features are considerable at wavelengths corresponding to the peak of the stellar spectral energy distribution.  \end{abstract}

\section{Introduction}

Precise radial velocity (\RV) measurements are critical to the discovery and characterization of exoplanets. 
On order of one dozen dedicated spectrographs exist for the purpose of \RV\ planet-hunting, with at least as many more currently under construction \citep{Wright2017}. 
However, significant challenges exist in deriving precise \RV\ measurements from these spectra. 
%The current capabilities of extreme-precision radial velocity (\EPRV) instruments do not extend to the 10 \cms\ regime, 
%In this work, we present an open-source code for \RV\ determination. We use a highly flexible linear model to extract \RV s in a fully data-driven way. %Our method simultaneously models the stellar and telluric spectra.

One of the primary drivers of the \RV\ noise budget is the incomplete treatment of telluric features in the
Earth's atmosphere \citep{Halverson2016}. 
Often, particular sections of a spectrum that are likely to feature telluric features are identified before the velocity shift of the stellar spectrum is inferred. 
These regions are then removed from analysis, leaving only telluric-free regions to be analyzed \citep[e.g.][]{AngladaEscude2012}.

Such an approach has two significant issues. 
The first is that removing sections of the spectrum removes significant regions of the spectrum that can be used to infer the stellar radial velocity. 
Many of the regions of significant telluric absorption lie in the red-optical and near-infrared, where there are abundant narrow spectral features that can be used to improve RV precision (cite).
This is especially true for M dwarfs, which peak in emitted energy at $\approx 1 \mu$m and have many narrow molecular absorption features in their photospheres \citep{Figueira2016}.
Eliminating large chunks of these spectra will therefore significantly inhibit our ability to detect planets around M dwarfs through RVs.

Secondly, not all telluric features are obvious. 
The Earth's atmosphere induces many small-amplitude features, often referred to as ``microtellurics,'' which are not obvious by eye but can affect the star's inferred RV at the $\sim 1$ \ms\ level \citep{Cunha2014}. 
As the locations of these features are not known \textit{a priori} and may not even be apparent in stacked spectra of many observations, these spectral regions cannot be thrown out. 
Instead, alternative methods to account for these features must be developed and employed in order to mitigate the effect of the Earth's atmosphere on the measured stellar radial velocities.

One such approach is modeling the telluric spectrum using existing line databases like HITRAN \citep{HITRAN2016}. 
The telluric model may then be divided out from the observations, assuming the line spread function of the instrument is known \citep[e.g.][]{Seifahrt2010}. 
This method relies on existing physical knowledge about the Earth's atmosphere and can be fine-tuned using local observatory measurements of e.g. atmospheric water vapor content \citep{Baker2017}. 
However, line databases are incomplete even in significant absorption features when compared to actual observations and certainly do not include microtellurics, making them poorly suited for extreme precision RV applications \citep{Bertaux2014}.

Another option is the use of telluric standard observations: a spectrum of a rapidly rotating early-type star, which is virtually featureless due to extreme rotational line broadening, may be used as a telluric model and divided out. 
This approach has the advantage of naturally reproducing the instrumental line profile and current observing conditions if the standard star has a line-of-sight vector sufficiently close to the target and if both observations are taken close together in time. 
For these conditions to be true, though, requires a significant investment of observing time, which planet search programs often cannot afford. 
Additionally, artifacts may remain near strong telluric features due to the imperfect correction of unresolved features \citep{Bailey2007}.

An alternative approach is the simultaneous modeling of both telluric and spectral features from the data. 
As the Earth's motion around the barycenter of the solar system induces a Doppler shift considerably larger than both the motion of telluric features and the size of a single pixel on the detector, these two spectra can be disentangled.
This process is well-established in the analysis of binary star systems through the development of linear models
\citep[e.g.][]{Simon1994} and in a Gaussian process framework \citep{Czekala2017}.
In these cases, both spectra are assumed to be unchanging in time, which is a reasonable approximation of a stellar spectrum but not necessarily of the telluric spectrum.
A more complicated linear model may be useful for separating the stellar and telluric spectra, enabling a reconstruction of both at each epoch.
\todo{(say something about Artigau2014 here)} 

\todo{[Other works that have done similar stuff -- be sure to cite: Artigau2018, Gao2016]}

Here we develop a linear data-driven model to infer a telluric and stellar spectrum and calculate the stellar RV at each observed epoch. 
The telluric model component may vary with time in a low-dimensional manner, which is also inferred from the data. 
Our model requires no prior knowledge of the star or the Earth's atmosphere. 
As such, it does not yield absolute measurements of \RV s, only highly precise relative measurements between epochs.

In this work, we focus on the ultra-stabilized spectrograph case, i.e. no absorption cell. 
We also assume that multiple epochs of observations exist and that these epochs are spread out across the observing season(s). 
This assumption is necessary to enable the disentangling of telluric features from the stellar spectrum. 
In this sense our pipeline is intended as a post-processing step, not a real-time data reduction service. 

In Section \ref{s:methods}, we outline the model and present an open-source implementation in \python\ and \TF\ called \wobble. 
In Section \ref{s:results}, we apply our method to \HARPS\ archival data for two target stars, the planet-hosting solar analog 51 Peg and the quiet M dwarf \Mdwarf, as a demonstration of \wobble's capabilities. 
\todo{We find xyz.} 
We conclude in Section \ref{s:future} with a detailed look at the limitations of the current implementation and outline ways to adapting \wobble\ for such cases as instruments with absorption cells, intrinsic time variability in the stellar spectrum, and multiple-star systems.

\section{Methods}
\label{s:methods}
\subsection{Model}

\todo{Our fundamental assumptions: single, unchanging stellar spectrum with \RV\ shifts; convolved telluric features which can vary in both \RV\ and shape. Instrument-provided wavelength solution which is good.}

Our data are the $M \times N$ matrix Y, where each entry $y_{m,n}$ is the observed ln(flux) for pixel $m$ of M at epoch $n$ of N. We also have a wavelength solution vector $\lambda$ of length M.

Our model is comprised of three major contributions to the observed spectrum: the star, which has a radial velocity about its center of mass $v_{\star}$ combined with motion of the Earth around the solar system barycenter; the atmospheric telluric spectrum, which has an effective radial velocity $v_t$ at each epoch; and the instrument, which has no velocity shift. Each of these components may vary with time in their spectral shape. 
We make the assumption that these variations can be captured in a low-dimensional space.

Each data column $y_n$ is modeled as:
$$y_n = P(v_{\star, n}) W_{\star} x_n + P(v_{t, n}) W_{t} z_n + W_{c} u_n + noise,$$
where $P$ is an interpolation operator which applies a Doppler shift $D_v$ and $W$ is a matrix of ``principal components'' which are weighted by $x_n, z_n, u_n$ to form the spectral contribution at epoch $n$. Each $W$ takes the shape $M' \times K$, where $M'$ is the length of a wavelength-space grid $\lambda'$ which may be different from the data wavelength grid $\lambda$ and $K$ is the number of components to be used (where $K_{\star}$, $K_t$, and $K_c$ may be different numbers).

\todo{discuss degeneracies in the model}

We build an initial template for both the stellar and telluric spectra by assuming a Doppler shift for each (equal to the barycentric correction and zero, respectively) and calculating the median flux at each model wavelength $\lambda'$ after that shift is applied.
With these models in hand, we are then able to infer an RV shift for both model components at each epoch.
We evaluate a likelihood for each epoch such that
$$ \ln \mathcal{L} = \sum_{m} -0.5 (y_m - y_m')^T C^{-1} (y_m-y_m'),
$$
comparing our model $y'$ to the data $y$, with $C$ representing our covariance matrix of the uncertainties on each data point.
Our model is agnostic to the intrinsic wavelength of any individual stellar or telluric feature, so while we infer the  Doppler shift relative to our model, it does not apply any information about the absolute \RV\ of the star relative to the solar system. 
To mitigate the potential degeneracy of developing a model effectively redshifted relative to all observations, we add (in log space) a Gaussian prior to our likelihood value on the mean \RV\ for all $N$ epochs such that
$$ P(m|X) = \ln \mathcal{L}  -0.5 * \frac{\sum(v)^2}{(Nl)^2}.
$$
\todo{someone should define these variables}

We maximize this likelihood function for our assumed stellar and telluric models iteratively, first calculating the best-fitting stellar \RV shifts holding the telluric shifts constant, then repeating for the telluric spectrum shift holding the stellar \RV s constant.
After each optimization, we use these velocities to improve our models for the telluric and stellar spectra, employing a gradient descent scheme to modify our templates $m'$ at each wavelength $\lambda'$ using the newly calculated velocities.
We repeat this procedure, calculating velocities for each epoch and in turn using these velocities to infer an improved model spectrum, until the procedure converges. 
\todo{(say something about model linearity and why this iterative procedure is effective)}

\subsection{Code}

The above-described model can be implemented in a variety of ways. We chose to build our code, \wobble, using the \TF\ package \todo{[terminology?]} in \code{python}. \TF\ is a \todo{[blah blah technical blah]}. While it has primarily been used in machine-learning contexts in the astronomical literature \todo{(cite)}, its auto-differentiation and fast linear algebra capabilities make it valuable for this use case as well. In particular, the necessary optimizations can be performed with high efficiency by \TF: the below-described analysis of \todo{x} spectra of 51 Peg runs in \todo{y} hours on a standard Macbook laptop.

\todo{(more details of implementation)}

In its current realization, \wobble\ is optimized for use with data products from the High Accuracy Radial Velocity Planet Searcher (\HARPS) spectrograph \citep{Mayor2003}. Like most \EPRV\ instruments, \HARPS\ is an echelle spectrograph spanning many spectral orders. These orders can be considered as independent spectra \todo{for reasons}. \todo{We run on each echelle order independently and then combine them like so:}


Our code is open-source and publicly available on GitHub.\footnote{\url{https://www.github.com/megbedell/wobble}}



\section{Results}
\label{s:results}
\subsection{51 Pegasi}

\todo{Description of the data, cite \acronym{ESO} archive.}

\todo{Show our results: we find the planet with the correct mass and orbital parameters. Compare to \HARPS\ pipeline.}

\todo{Plots of RV vs. time, models of star and tellurics in a few segments of the spectrum.}

\subsection{\Mdwarf}

\todo{Results \& more plots.}

Compare to HARPS-TERRA results for this star \citep{AngladaEscude2012}.

\section{Generalizing \wobble}
\label{s:future}

\todo{Revisit the assumptions: single star, no gas cell, known instrumental wavelength solution, no intrinsic spectral variability, no additive terms, many spectra exist. 
In what situations are these assumptions violated, and how would we configure \wobble\ to deal with them?}

Intrinsic stellar variability: cite \citet{Davis2017} for PCA approach

\bibliographystyle{apj}
\bibliography{paper.bib}%general,myref,inprep}

\end{document}  